\documentclass{report}
\usepackage[utf8]{inputenc}
\usepackage{graphics}
\graphicspath{ {./images/} }
\usepackage{multicol}
\usepackage{parskip}

% width,height
\usepackage[a4paper, total={7in, 10in}]{geometry}

\title{IOT- Research Paper Summary}
\author{Rhea Pandita}
\date{\today}

\begin{document}

    \begin{titlepage}
    \centering
        \vspace*{2cm}
        \Huge
        \textbf{Edge Networks and Devices for the Internet of Things}
        
        \vspace*{0.6cm}
        \Large
        \textit{Research Paper Summary}
        
        \normalsize
        \vspace*{1.5cm}
        Rhea Pandita\\
        \vspace{0.2cm}
        21011101100\\
        \vspace{0.2cm}
        AI-DS B\\
        
        \vfill
        
        %\large
        
        %\begin{figure}[h]   % Figure Environment
        %    \centering
        %    \includegraphics[width = 6cm]{logo2}  % including the picture
        %    \caption{Muhammad Waseem H}
        %    \label{fig:my_image}
        %\end{figure}
        
       
        
        \vfill
        
        %\includegraphics[scale=0.5]{logo}
        
       
        Computer Science and Engineering\\
        Shiv Nadar University, Chennai\\
        21st January 2023
        \vspace*{1cm}
    
\end{titlepage}


    
\setlength{\columnsep}{1.0cm}
    \large
    \section*{Summary}
    Internet of Things has been a growing field in the past few years. It has been a convenience to automate multiple devices from an exterior area to transfer data related to the device without having to transit to a location. This, has been made possible by IOT. But, IOT did not happen to occur to a person out of nowhere. Internet paved the way for it. The research paper considers how existing concepts underlying the development of the internet are being strained in IOT. The similarities between the development and growth o the Internet and IOT are the main highlights.

    
    The concept of the Internet was simple compared to IOT. It was easier to persuade suppliers to adopt a universal protocol for connecting computers together because it had a straightforward concept that easy to understand and deal with.

    
    For IOT, large industrial and political names had to be persuaded to adopt a common approach. Due to the increase in the number of devices in use, the magnitude and scalability posed a challenge. Implementing security measures became all the more vital. But, the knowledge of the deployment of the Internet dictated what is required for advancing in IOT.
    
    Since IOT deals more with the configuration process, application design and implementation, more emphasis is put on the development of devices than in the deployment. The need for scaling up to larger no of devices for more people with trusted security authentication and authorization is to be looked at while also making sure that flexibility is induced with growing times. Facilitation of such processes is done by developing on a virtual deployment, adopting structure of Domain Name Service, utilizing the Identifier Management System and most importantly the growth of cyber computing.
    
\begin{multicols}{1}    
    \section*{Key points from the author}
    %our understanding on what the authors have contributed or raised as comments regarding the topic.
     
    \begin{enumerate}
        \item While Internet was in its early stages it was necessary to link together many network deployments of different architectures.Adopting a universal network type was chosen instead of trying to adapt each network type available for suitable needs.
        \item It was imminent that no.of machines produced would be inadequate in contrast to the increase in population. Therefore, it was imperative to introduce private address spaces that could be reused and for devices to have more than one address.
        \item Introduction of a new protocol IPv6 resolved the addressing problem by allowing multiple addresses for a single interface,group operations, limiting the scopes of addresses, and improvements in mobility support, among other gains.
        \item Domain Name Service (DNS) was introduced to connect user-friendly names to Internet addresses to help the layman learn the directory of computers better.
         \item DNS was modified wrt 3 important aspects : 1)adding descriptions of services, the ability to search for services and capability to authenticate DNS entries.
         \item With the move towards IOT, the scope of edge devices broadened (printers, telephones etc). However, digital controllers in these devices all obey the Internet protocols.
         \item Different forms of deployment exist in IOT, including - producing data as a result of interactions from an application or spontaneously and continually.
         \item It is much easier to concentrate on DOs or Digital Objects, an identifier of the "type-value" pair to represent objects than using huge descriptions of devices, network types, deployment, applications and data.
          \item It is really appreciative if a general template is stored for applications rather than creating new ones for specific applications detailing the deployment.
    \end{enumerate}
\end{multicols}
    
    \section*{My views about this paper}
    %Your views on the topic, along with what you envision the future of AI would be in the topic discussed in the paper.
    In my opinion, the deployment of physical devices that can be connected to the internet and used for communication is what lead to inclination of IOT development. I agree with the points written by the author in regards to which aspects of the Internet have lead to the ascent of IOT. Changes made to scale a large host of devices, improving authentication and authorization properties and flexibility for multiple communities to adapt to working with this has been the recipe of success. Added benefits of technology advancements for the maintenance of such devices and it data improve the quality further.Adopting the DNS structure indeed is helpful for managing scalability and growth of the cloud computing domain can expand the scope further.
       

\end{document}

